\documentclass{article}

\usepackage[utf8]{inputenc}
\usepackage{graphicx}
\usepackage{amsmath}
\usepackage{amssymb}

\usepackage{mavlab}
\usepackage{times}

\title{MAVLab style guide}
\author{E.J.J. Smeur\thanks{Email address: e.j.j.smeur@tudelft.nl}, A.B. Author, C.D. Writer \\ Delft University of Technology, Kluyverweg 1, Netherlands}
\begin{document}
\maketitle
\begin{abstract}
	The abstract is in bold font.
	Limit the text here to 150 words.
	\end{abstract}

\section{Introduction}
You can use this style for your paper writing endeavours, and for self-archiving, which is often not allowed with the journal formatting.

If you have any improvements, please discuss and implement them at: \url{https://github.com/tudelft/latex_style}.

\section{Citing}
For citing papers, use \textbackslash cite\{\} whenever you want to use the citation in the sentence.
Example: \cite{smeur} investigated INDI for the attitude control of MAVs.

Alternatively, you can cite a paper between parentheses with \textbackslash citep\{\}.
Example: Behaviour trees can be used to evolve robot behaviour \citep{Scheper2016a}.

\section{Sections}

You can use sections to separate big parts of your paper.

\subsection{Subsection}

For further specification use subsections.

\subsubsection{Subsubsection}
And lastly subsubsections.

\section{Notation of refs}

Capitalize the first letter of the words Equation and Figure when you follow with a \textbackslash ref.
Example: Equation \ref{eq:example} is a very interesting equation.

\begin{equation}
	1 + 1 = 2
	\label{eq:example}
\end{equation}

\section*{Acknowledgements}
This section does not get a number.

\bibliography{bibliography}

\end{document}
